\selectlanguage{greek}
\section*{Περίληψη}

Η παρούσα διπλωματική εργασία αποτελεί μια προσέγγιση από πρώτες αρχές στη διαδικασία ταξινόμησης της νόσου Αλτσχάιμερ και της άνοιας με τη χρήση τεχνητής νοημοσύνης. Η νόσος του Alzheimer και η άνοια είναι από τις πιο διαδεδομένες παθήσεις στους ηλικιωμένους, με καταστροφικές οικονομικές συνέπειες , λόγω του κόστους περίθαλψης κατά τη διάρκεια της πάθησης . Οι προβλέψεις υποδεικνύουν συνεχή αύξηση των διαγνώσεων καθώς ο παγκόσμιος πληθυσμός συνεχίζει να γερνάει ραγδαία. Στόχος αυτής της εργασίας είναι η ανάλυση κάθε σταδίου της κλινικής διαδικασίας—από την παθοφυσιολογία και τους βιοδείκτες που μπορούν να χρησιμοποιηθούν έως τη φυσική των τεχνικών απεικόνισης και τις προσεγγίσεις μηχανικής μάθησης που χρησιμοποιούνται στην προεπεξεργασία και την ταξινόμηση. Ο στόχος είναι η ανάλυση των τρεχουσών πρακτικών και η παροχή μιας ανάλυσης βασισμένης στην πιο πρόσφατη έρευνα. Επίσης, εξετάζουμε βασικές προκλήσεις και περιορισμούς, ιδιαίτερα την ανισορροπία κλάσεων στα ιατρικά σύνολα δεδομένων, και περιλαμβάνουμε πειράματα κώδικα που συγκρίνουν συνελικτικά νευρωνικά δικτυα και άλλες τεχνικές για την επιβεβαίωση και την προσθήκη αξιοπιστίας στην έρευνα και τα ευρήματα μας. 


\vspace{1cm}
\hrulefill
\\
\textbf{Λέξεις Κλειδιά:} Νόσος Aλτσχάιμερ, Άνοια, Βαθιά Μάθηση, Νευροαπεικόνιση, Ταξινόμηση
\selectlanguage{english}

\clearpage
