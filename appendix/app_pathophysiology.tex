\chapter{Pathophysiology of Alzheimer}
\begin{enumerate}
    \item \label{Appendix-1} Amyloid-$\beta$ (A$\beta$) pathology is generally thought to begin with the early deposition of A$\beta$42, a more aggregation-prone and fibrillogenic isoform of the peptide. As these deposits accumulate, they disrupt the surrounding neuronal environment and ultimately influence the stability and function of the tau protein network, which is essential for maintaining axonal structure and intracellular transport. The subsequent tau dysfunction and formation of neurofibrillary tangles contribute directly to progressive neuronal degeneration characteristic of Alzheimer’s disease. Although the preferential production or accumulation of A$\beta$42 over other forms such as A$\beta$40 can be influenced by genetic factors, no single mechanism fully accounts for this shift. It is also important to note that many individuals produce A$\beta$42 without developing Alzheimer’s disease; pathology emerges when the peptide accumulates beyond the brain’s capacity for clearance, leading to plaque formation. \cite{hendersonNeurobiologyAlzheimersDisease1989}
\end{enumerate}


