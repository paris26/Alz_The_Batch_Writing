
\chapter{Overview of Dementia and Alzheimer's disease imaging}
 
\section{Magnetic Resonance Imaging (MRI)}

Magnetic Resonance Imaging (MRI) is a technique based on NMR principles. Even though the classical view is not fully correct, it is quite useful as an introduction to the technique. Atomic nuclei possess the property of spin, which can be imagined as a spinning top with that property being its angular momentum. When a large magnetic field is applied to the nuclei, they tend to align with the field while continuing to spin.
Unlike other modalities such as CT scans, which can subject the patient to ionizing radiation, MRI does not cause any exposure-related hazards since it uses non-ionizing electromagnetic radiation \cite{davisNuclearMagneticResonance1982}. Excitation in the case of MRI happens through radio-frequency (RF) pulses specifically tuned to the Larmor frequency. At this frequency, the nuclei resonantly absorb energy, causing the net magnetization vector to tip away from the applied magnetic field. The process analyzed classically is analogous to an oscillating body that performs a periodic movement (forced oscillator), receiving energy at a matching frequency to enhance the amplitude of its motion (e.g., a swing that gets pushed at the right intervals). It should be noted that pulse sequences can vary, but more detail will be discussed later. Once the pulse sequence stops, the nuclei relax back toward equilibrium and while doing so they emit a detectable signal captured by receiver coils \cite{kattiMagneticResonanceImaging2011, plewesPhysicsMRIPrimer2012, davisNuclearMagneticResonance1982, paiMagneticResonanceImaging2025}.
Capturing, however, does not happen all at once. To achieve localized imaging of tissue, magnetic field gradients are employed. They alter the magnetic field so that it varies spatially, and in doing so only selected tissue precesses at the frequency matching the RF pulse.
Overall, MRI offers a non-invasive way that possesses great accuracy in capturing the underlying tissue.

\subsection{Spin}

Atomic nuclei possess an intrinsic property known as \textit{spin}. Spin occurs in multiples of $1/2$ and may be positive or negative. Not all particles possess spin; it appears in particles with an odd number of protons and neutrons ($P+N=\text{odd}$) \cite{plewesPhysicsMRIPrimer2012}.

\subsection{Properties of Spin}

When a particle with net spin is placed within a magnetic field of strength $B$, it can absorb a photon of frequency $\nu$. This frequency depends on the gyromagnetic ratio $\gamma$ of the particle \cite{vangeunsBasicPrinciplesMagnetic1999} :

\[
\nu = \gamma B .
\]

For hydrogen atoms, the gyromagnetic ratio is \cite{kattiMagneticResonanceImaging2011}

\[
\gamma = 42.58\ \text{MHz/T}.
\]

MRI uses nuclei with either an unpaired proton or an unpaired neutron, allowing them to possess a net spinning charge---i.e., angular momentum. Because spin is associated with electric charge, these nuclei generate a magnetic field and behave as magnetic dipoles.

\subsection{Hydrogen Nuclei in MRI}

Hydrogen nuclei  are the most abundant MR-visible nuclei in the human body (primarily in water, H\textsubscript{2}O) and consist of a single proton with non-zero spin, allowing them to act as magnetic dipoles under a strong magnetic field \cite{plewesPhysicsMRIPrimer2012,groverMagneticResonanceImaging2015}. When a magnetic field is applied, all hydrogen spin angular momentum vectors tend to align either parallel (spin-up) or anti-parallel (spin-down) to the field. Since spin projection comes in quantized these energy states are distinct. Nuclei can transition between different energy states by absorbing or emitting energy , which in the case of MRI is provided by RF pulses at the ( Larmor ) resonance frequency. 


The axes of spinning protons do not align perfectly parallel to the applied field; rather, they \textit{precess}. The frequency of precession is called the \textit{Larmor frequency}, which is proportional to the magnetic field strength:

\[
\omega = \gamma B .
\]

After applying the magnetic field, the sum of all nuclei yields a weak net magnetic moment or magnetization vector (MV) parallel to the field. Applying an RF pulse at the Larmor frequency in the presence of a magnetic gradient selects a slice of tissue and excites the nuclei, causing them to absorb energy and rotate into the plane of the RF pulse.

Longer RF pulses produce greater rotation angles. If the pulse has sufficient intensity, it flips the magnetization vector into the transverse (XY) plane, creating a \textit{90° flip angle} in which all protons precess in phase.

At this moment, an RF signal is induced in a receiver coil. This signal depends on the presence and molecular environment of hydrogen atoms. Tightly bound hydrogen, such as in bone, produces weak or no usable signal, whereas loosely bound hydrogen in soft tissues produces a strong signal. The concentration of loosely bound hydrogen is known as the \textit{proton density} or \textit{spin density}.

\subsection{Relaxation}

When the RF pulse is turned off:

\begin{itemize}
    \item The excited nuclei return to their lower-energy spin state, emitting energy. This decay of transverse magnetization is detected as the \textit{free induction decay} (FID).
    \item Nuclei realign with the main magnetic field through energy transfer to surrounding molecules. This is the basis of \textit{longitudinal} or \textit{spin-lattice} relaxation.
\end{itemize}

The time constant describing the recovery of longitudinal magnetization is called the \textit{T1 relaxation time}. Typical values are approximately 500 ms for short T1 tissues. The decay of transverse magnetization, governed by loss of phase coherence among precessing protons, is described by the \textit{T2 relaxation time} (e.g., $\sim 80$ ms for many tissues).

\subsection{T1- and T2-Weighted Imaging}

Relaxation properties allow distinctions between tissue types. MRI commonly uses two contrast mechanisms:

\paragraph{T1-weighted imaging} uses short repetition times(TR) and is detecting T1 relaxation. Tissues that appear bright have short T1 values (e.g. fat) , while those with longer T1 values appear darker . This modality is specifically useful for visualizing anatomical detail due to its characteristic of high spatial resolution.

\paragraph{T2-Weghted Imaging} uses long repetition times (TR) and long echo times (TE) . Tissues that are water rich or are abnormal usually have long T2 values and appear bright. \cite{kattiMagneticResonanceImaging2011}

By changing TR, TE values and the spatial localization of excitation using gradient fiels, MRI can provide great anatomical detail while maintaining sensitivity to underlying pathological changes. These contrast mechansims are the foundation of MRI and its use in clinical settings. 

\section{The Clinical Use of Structural MRI in Alzheimer Disease}

In the case of Alzheimer’s disease, we use the features of structural imaging that MRI can produce, and it has become an integral part of the clinical assessment process. New data from clinical studies showing changes in structural markers from preclinical to overt stages of the disease are reshaping the diagnostic landscape and influencing future diagnosis and treatment.

In AD Alzhimer can help find structural changes that take place in the brain of a patient , and it is has become routine , with MRI being one of the most commonly used modalities among clinical settings . As has been mentioned before structural brain changes can take place even up to 20 years before without cognitive decline symptoms and become more obvious and accelerate decline as time goes on. One of the earliest and important signs is shrinkage in the medial temporal areas of the brain, which is used to now diagnose MCI (Mild Cognitive Impairment) . MRI can also allow to distinguish between AD and other types of dementia. Aditionally , shrinkage in the hippocampus , or in the brain matter as a whole can be a sign that neurodegeneration is already underway , marking an important step in the progress of the disease.

MRI is also increasingly used in research to track the effectiveness of disease-modifying drugs based on the structural markers mentioned above. While MRI plays an important role in tracking disease progression and determining atrophy in specific brain regions, it is essential to point out that many other imaging and non-imaging techniques are used in clinical assessment.

The critical utility of MRI in the clinical domain has been discussed, particularly for AD and dementia. By adopting standardization protocols during acquisition or as part of further processing after the capture of a scan, we can ensure robustness and reliability. Any subsequent algorithmic task will thus be further improved \cite{frisoniClinicalUseStructural2010}.
The pattern of AD development specifically follows a progressive accumulation of abnormal proteins (e.g., A$\beta$42), which in turn lead to synaptic, neuronal, and axonal damage. This pattern of accumulation occurs many years before any cognitive dysfunction becomes evident in patients. The progression in AD follows quite a typical pattern. It first presents itself in the medial temporal lobes (entorhinal cortex and hippocampus), which then spreads to inflict further neocortical damage \cite{braakNeuropathologicalStageingAlzheimerrelated1991, delacourteBiochemicalPathwayNeurofibrillary1999}.
This delay, in fact, suggests that the toxic effects induced by abnormal protein deposition have to reach a certain threshold to initiate noticeable cognitive symptoms. For example, in amnesic Mild Cognitive Impairment (aMCI), deficits must emerge across multiple cognitive domains before the full AD diagnostic criteria are fulfilled.
Additionally, the development of disease-modifying drugs that can decrease the rate of decline—or potentially halt it—requires early identification of at-risk individuals to evaluate treatment efficacy. Consequently, the value of early diagnosis has increased substantially.
Studies have already shown that significant correlations exist between damage in regions such as the hippocampus and entorhinal cortex and the likelihood of MCI-to-AD progression. This constitutes the primary objective of this work: evaluating whether algorithmic approaches can reliably predict this conversion, which would have significant implications for treatment planning and disease management \cite{frisoniClinicalUseStructural2010}.

\section{Atrophy as a Neurodegeneration Marker}

Atrophy is a consequence of progressive neurodegeneration. Structural changes can be mapped onto the stages of tangle deposition based on the Braak staging system, as well as onto specific neuropsychological deficits. The earliest signs occur in the perforant pathway, producing memory impairments. Later changes in the parietal and frontal neocortices correspond to language deficits, visuospatial impairments, and behavioral changes.

Changes in structural measures such as whole-brain volume, entorhinal and hippocampal volume, temporal lobe volume, and ventricular enlargement can be identified using MRI, and these measures may serve as potential biomarkers. To be used clinically, such biomarkers must be identifiable across all patients and provide clear distinctions between disease stages.

In the asymptomatic stage, amyloid markers provide indirect evidence of disease pathology, whereas after the onset of Mild Cognitive Impairment (MCI), structural changes are more sensitive indicators \cite{frisoniClinicalUseStructural2010}.

\section{Alzheimer Disease Criteria and MRI}

The diagnostic approach to AD as proposed in the International Working Group (IWG-2) Criteria has shifted from a clinicopathological to a clinicobiological one. This shift reflects the incorporation of biomarkers into the clinical diagnosis process. The new framework requires two criteria to be satisfied: an appropriate clinical phenotype (typical or atypical) and the presence of a biomarker consistent with AD pathology.

The criteria are complete and can capture all disease stages, including typical AD, atypical variants, mixed AD, and preclinical states (asymptomatic at-risk and presymptomatic AD).

Pathophysiological markers are restricted to those that directly indicate amyloid or tau pathology and must be present for a diagnosis of AD.

\subsection*{Specific Criteria}

\begin{itemize}
	\item A CSF profile showing decreased A$\beta_{42}$ alongside increased T-tau or P-tau concentrations.
	\item Elevated tracer retention on amyloid PET imaging.
	\item Presence of an autosomal dominant AD mutation.
\end{itemize}

Topographical biomarkers such as volumetric MRI (e.g., hippocampal atrophy) and FDG-PET have been removed from the core diagnostic algorithm, as they lack sufficient pathological specificity for AD detection. Their new role is to monitor disease progression over time \cite{duboisAdvancingResearchDiagnostic2014}.

Despite this change, hippocampal atrophy remains one of the most established and validated MRI markers of AD. In vivo measurements correlate with Braak staging and neuronal counts, with volume reductions varying by disease stage (e.g., 15--30\% in mild dementia, 10--15\% in MCI with mild dementia) \cite{frisoniClinicalUseStructural2010}.

\section{Computed Tomography (CT)}

CT scans are a useful tool in the diagnosis of Alzheimer's disease (AD), providing images of anatomical structures in brain regions such as the medial temporal lobe, where atrophy is considered a marker for AD conversion. However, during the diagnostic process of cognitive complaints or deficiencies, MRI scans are generally preferred due to their superior soft-tissue contrast. CT scans are used primarily when MRI presents contraindications (e.g., pacemakers)~\cite{ferreiraNeuroimagingAlzheimersDisease2011}.

The use of CT in combination with nuclear imaging techniques has increased, first with the introduction of PET/CT and later SPECT/CT in AD diagnosis. These combined modalities emerged because the anatomical localization of functional abnormalities in nuclear imaging alone was often imprecise. Adding CT provides accurate anatomical reference and resolves localization issues, provided the modalities are properly coregistered~\cite{townsendPETCTToday2004}.

Given this context, the review of the basic principles of CT will remain brief, since this work does not focus on CT images directly, nor do CT scans provide structural detail comparable to MRI. Nevertheless, CT remains an important imaging method and should be acknowledged.

\subsection{Mechanics of a CT Scan}

\subsection{Historical Development of CT}

The discovery of X-ray radiation is attributed to Wilhelm Conrad Röntgen, who performed the first cathode-tube experiments on November 8, 1895. X-rays possess special properties: they can energize atoms (producing photons via fluorescence), and they can penetrate opaque materials.

Initially, X-ray projection imaging captured a two-dimensional image of a three-dimensional object. However, important structural information was lost due to overlapping tissues, while low-contrast regions were difficult to distinguish. Additionally, scattering produced noise and degraded image quality.

The term \textit{computed tomography} reflects its two essential components: ``computed,'' referring to numerical reconstruction, and ``tomography,'' meaning cross-sectional slicing. Modern CT scanners use X-ray energies between 100--150~kV.

Image reconstruction in CT relies on the mathematical foundations of the Radon transform and its inverse. Radon demonstrated that a 2D image can be reconstructed from a set of projections collected at multiple rotation angles.

\begin{figure}[htbp]
    \centering
    \includegraphics[width=0.75\linewidth]{images/pmp-32-1-1-f1.png}
    \caption{(a) The first CT image produced at Atkinson Hospital. (b) A modern CT scan from an advanced scanner \cite{jungBasicPhysicalPrinciples2021}.}
    \label{fig:ct_history}
\end{figure}

\subsection{Basic Physical Principles of CT}

\subsubsection{X-ray Attenuation}

As X-rays pass through tissue, some photons are absorbed or scattered 
while others reach the detector. Different tissue types attenuate 
X-rays differently—bone absorbs much more than soft tissue, for instance. 
This selective absorption is what creates contrast in the final image. 
The phenomenon is described by the exponential attenuation law:

\begin{equation}
	I_{x} = I_{0} \, e^{-\mu x},
\end{equation}

\begin{itemize}
	\item $I_0$: initial intensity
	\item $I_x$: transmitted intensity at the detector
	\item $\mu$: linear attenuation coefficient (tissue-dependent)
	\item $x$: material thickness
\end{itemize}

For multiple materials along the path:
\begin{equation}
    I = I_{0} \exp\left[-(\mu_{1}x_{1} + \mu_{2}x_{2} + \cdots)\right],
\end{equation}
known as the Lambert–Beer law.

Attenuation can also be expressed as a line integral:
\begin{equation}
    \ln\left(\frac{I}{I_{0}}\right) = -\int \mu(s)\, ds.
\end{equation}

\begin{figure}[htbp]
    \centering
    \includegraphics[width=0.75\linewidth]{images/pmp-32-1-1-f2.png}
    \caption{Illustration of X-ray attenuation through a sample composed of multiple materials. The final intensity corresponds to the cumulative effect of all attenuation coefficients along the beam path.}
    \label{fig:xray_attenuation}
\end{figure}

\subsection{Data Acquisition}

The process of data acquisition in a CT scan begins by placing the patient or sample inside the scanner. A rotating X-ray source then emits radiation as it orbits the sample, while a detector on the opposite side captures the transmitted signal. These signal intensities are digitized by the Data Acquisition System (DAS) and prepared for reconstruction.

Acquisition criteria:

\begin{itemize}
    \item Projections must be collected over many angles (typically 360° or 180° with symmetry).
    \item Each projection must fully include the object.
    \item The object must remain still.
\end{itemize}

\subsection{Image Reconstruction}

The goal of CT reconstruction is to compute the 2D attenuation map from the measured 1D projections (line integrals).

For an object described by a function $f(x,y)$, the Radon transform is:
\begin{equation}
    p(s, \varphi) = \mathcal{R}f(x,y).
\end{equation}

To convert $(x,y)$ to rotated coordinates $(s,u)$:
\begin{equation}
\begin{pmatrix}
x \\ y
\end{pmatrix}
=
\begin{pmatrix}
\cos\varphi & -\sin\varphi \\
\sin\varphi & \cos\varphi
\end{pmatrix}
\begin{pmatrix}
s \\ u
\end{pmatrix}.
\end{equation}

Thus, the projection data are given by:
\begin{equation}
    p(s,\varphi) = \int f(s\cos\varphi - u\sin\varphi,\; s\sin\varphi + u\cos\varphi)\, du.
\end{equation}

Early CT systems applied a high-pass filter to sharpen images. Today, reconstruction methods include:

\begin{itemize}
    \item matrix inversion,
    \item iterative algorithms,
    \item Fourier techniques,
    \item filtered backprojection (FBP),
    \item 3D Radon-based approaches.
\end{itemize}

The most common modern method is based on the \textit{central slice theorem}, which links the Radon transform to the 2D Fourier transform.

\subsection{CT Numbers / Hounsfield Units}

A CT image consists of voxels (3D volume elements). Each pixel’s intensity reflects the mean attenuation coefficient in its voxel. CT uses a 12-bit grayscale (4096 levels). The Hounsfield Unit (HU) scale normalizes attenuation values relative to water:

\begin{equation}
    \text{HU} = 1000 \cdot \frac{\mu_{\text{pixel}} - \mu_{\text{water}}}{\mu_{\text{water}}}.
\end{equation}

\begin{figure}[htbp]
    \centering
    \includegraphics[width=0.7\linewidth]{images/pmp-32-1-1-f4.png}
    \caption{Hounsfield scale and representative values for different tissues \cite{jungBasicPhysicalPrinciples2021}.}
    \label{fig:hounsfield_scale}
\end{figure}


\section{PET / PET-CT}

Positron Emission Tomography (PET) is an analytical process in which compounds labeled with radioisotopes are used as molecular probes to image and measure biochemical processes \textit{in vivo}.

\subsection{Basic Physics}

Positron emission is based on proton-rich nuclei, called ``emitters,'' which are unstable. To stabilize themselves, they may either release excess protons and gain neutrons, or capture electrons. The first process is known as \textit{positron emission}, and the second as \textit{electron capture (EC)}. Both processes are isobaric decays, meaning the mass number remains unchanged between the parent and daughter nuclei. In nuclei with low atomic weight, positron emission is more prevalent, while in heavier nuclei electron capture dominates.

It can be described by :

\[
_{m}X^{n} \longrightarrow {}_{m-1}Y^{n} + \beta^{+} + \nu
\]

\[
p \rightarrow n + e^{+} + \nu_{c}
\]

The nucleus by having too many protons , undergoes beta-plus decay . In this process a proton converts into a neutron, emitting a positron and a neutrino. This results in a new element with ; 
\begin{itemize}
	\item Same mass number ( protons + neutrons unchanged)
	\item Atomic Number decreased by one ( one fewer proton )
\end{itemize}

The neutrino leaves without interacting with any of the tissue, whereas the positron, being charged is slowed by scattering in the tissue. 

The distance traveled (the \textit{range}) depends on its energy:

\[
E_{\text{positron}} + E_{\text{neutrino}} = \text{transition energy} - 1022 \ \text{MeV}
\]

As the positron slows, it eventually annihilates with an electron, producing either:
\begin{itemize}
    \item a pair of 511~keV photons (direct annihilation), or
    \item a short-lived positronium, which also decays into two 511~keV photons.
\end{itemize}

The photons that are produced are emmited nearly $180^\circ$  apart. However due to residual momentum , the emission angle varies slightly around $180^\circ$ . Their energy is equal to that of a positron and electron at rest mass while their are more officially named "\textbf{annihilation photons} .

\subsection{Detection of Annihilation Radiation}

To detect the anihilation process the system identifies events through a coincidence detection window ( also known as electronic collmination ). Ths process works by creating a timing window (3-15ns) within which the photons need to strike two opposing detectors. If both of them arrive within this interval , the system records the interaction that occured , somewhere along the line of response ( LOR ) , which is the direct path connecting the two detectors. 

\subsection{PET Spatial Resolution Limitations}

\begin{enumerate}
    \item \textbf{Positron Range:}  
    Because the positron travels a finite distance before annihilation, the detected photons do not originate exactly where the positron was emitted. This introduces an intrinsic spatial uncertainty.

    \item \textbf{Non-colinearity of Annihilation Photons:}  
    The photons are not emitted at exactly $180^\circ$. Small deviations (typically $<0.5^\circ$) cause blurring. The effect depends on detector ring diameter—about 1 mm for a 50 cm ring and 2 mm for a 90 cm whole-body system.

    \item \textbf{Detector Size:}  
    The intrinsic spatial resolution depends on the size of the individual detector crystals used in modern PET scanners, which typically consist of small scintillator arrays coupled to larger photodetectors \cite{basuFundamentalsPETPET2011}.
\end{enumerate}

\subsection{PET-CT}

\subsubsection{PET and Anatomic Imaging}

A major limitation of PET imaging is its low spatial resolution and lack of anatomical detail, making it difficult to accurately localize lesions with abnormal radiotracer uptake. Distinguishing physiological from pathological uptake can be challenging.

Initially, non-integrated scanners were used: PET and CT images were acquired separately and then aligned visually or using fusion algorithms. However, differences in patient position and organ motion significantly reduced accuracy. The dual-scan approach is also costly, time-consuming, and uncomfortable for patients.

\subsubsection{Integrated PET/CT}

PET scanner was introduced in 1998. It has the ability to show both body structure through the CT scan but also body function by utilizing the PET scan. This not only helps doctors make more accurate decisions but also helps patients to make the process faster , easier and cheaper for them. \cite{townsendPETCTToday2004, basuFundamentalsPETPET2011}.

\subsection{The Use of PET in Alzheimer's Disease}

The pathology of AD is defined by amyloid plagues in the first place ( according to Braak staging) , tau tangles ( neurofibrillary tangles ) concentration , activated microglia, neurotransmitter changes, and neuronal loss.

The characteristic pathology of Alzheimer’s Disease (AD) includes the accumulation of amyloid plaques in the brain and intracellular hyperphosphorylated tau protein in the form of neurofibrillary tangles; activation of microglial cells; alterations in neurotransmitter levels; and neuronal death. Changes in the brain can occur for a long time prior to the onset of clinical manifestations. Therefore, it is now important to consider CSF analysis and brain imaging as early biomarkers and useful tools for tracking AD progression. \cite{duboisResearchCriteriaDiagnosis2007, duboisAdvancingResearchDiagnostic2014}.

New PET imaging technologies allow the identification of AD in prodromal stages and support the development and monitoring of disease-modifying treatments. PET enables measurement of multiple functional processes in the brain, including metabolism and neurotransmitter activity, providing region-specific insights that strengthen diagnostic accuracy and treatment evaluation.

\subsection{Processes Assessed by PET}

\subsubsection{Brain Glucose Metabolism}

To measure glucose metabolism, PET uses the tracer 2-fluoro-2-deoxy-glucose (F-FDG). A decline in glucose metabolism occurs long before clinical symptoms appear. This decline is region-specific, most prominently affecting the parietotemporal, frontal, and posterior cortices \cite{mosconiBrainGlucoseMetabolism2005}.

Patients with AD can be diagnosed with up to 90\% sensitivity using FDG-PET \cite{smallApolipoproteinType41995}. Differentiation from other dementias is more difficult.

Longitudinal PET studies show:
\begin{itemize}
    \item Conversion from healthy to MCI is best predicted by \textbf{medial temporal} glucose hypometabolism.
    \item Conversion from MCI to AD is best predicted by hypometabolism in the \textbf{posterior cingulate cortex}.
\end{itemize}

Functional MRI (fMRI) also helps distinguish patterns of brain activity. Its basis lies in the magnetic properties of deoxyhemoglobin and the fact that blood flow increases more than oxygen metabolism during neural activation. This produces a subtle MR signal increase known as the \textit{blood oxygenation level-dependent (BOLD)} effect. Modern fMRI achieves spatial resolution near 1~mm and temporal resolution around 1~s \cite{nordbergUsePETAlzheimer2010}.

