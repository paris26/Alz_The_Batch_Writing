\chapter{Introduction}

\section{Contextual Background}
The evolution of neuroimaging has transformed the diagnostic landscape for Alzheimer's disease. Multiple imaging techniques, including MRI, PET, and CT, are used to support clinical evaluation.

Structural Magnetic Resonance Imaging (MRI) excels at quantifying brain atrophy, particularly in the hippocampus, where volumetric reductions serve as predictors of progression from Mild Cognitive Impairment (MCI) to AD.

Positron Emission Tomography (PET), with amyloid- and tau-specific tracers, enables visualization of amyloid-$\beta$ (A$\beta$) plaques and tau pathology (for more on AD pathology, see Appendix~\ref{Appendix-1}). Multimodal approaches integrating multiple imaging modalities provide superior sensitivity and specificity for early diagnosis and longitudinal disease tracking. AI- and deep-learning–enhanced multimodal frameworks now fuse neuroimaging data with genetic, clinical, and neuropsychological variables, achieving diagnostic accuracies exceeding 95\% in diverse cohorts \cite{marquezNeuroimagingBiomarkersAlzheimers2019}.

%Neuroimaging offers objective and quantifiable biomarkers that surpass the limitations of subjective clinical assessment, enabling detection of subtle microstructural changes. Evidence shows that multimodal imaging improves differential diagnosis by monitoring atrophy, hypometabolism, and connectivity loss, while guiding treatment evaluation—critical in an era of emerging disease-modifying therapies \cite{kantarciNeuroimagingAlzheimerDisease2003}.%


Neuroimaging allows for a non-invasive way to produce quantifiable and objective measurements. The combination of multiple modalities can help to detect structural changes, hypometabolism , connectivity loss and even tau or amyloid deposition. At a time where new treatments emerge at the preclincal stage the use of neuroimaging can provide objective information about the progression stages and the efficacy of the treatment. \cite{kantarciNeuroimagingAlzheimerDisease2003}

The latest innovations in imaging technologies and the impact on the diagnosis of AD have shifted diagnoistic criterial. According to the Naitonal Insitute of Aging (NIA) and the Alzheimer's Association staging is based on amyloid , teu and neurodegeneration profiles.  Core-1 biomarkers include low CSF A$\beta$42/40 ratios or positive amyloid PET, with Core-2 Biomarkers tracking progression  - elevated tau PET or plasma p-tau217) . MRI protocols like diffusion tensor imaging and quantitative susceptibility mapping are used for enthorhinal-hippocampla circuits where early tau deposition can begin even up to two decades early. \cite{kantarciNeuroimagingAlzheimerDisease2003}

For an expanded overview of AD pathology, see Appendix~\ref{Appendix-1}.

AD develops slilently for 15-20 years, making early identification and diagnosis crucial. The biomarkers of tau PET and plasma p-tau217 trace early tau sprad and predict neurodegeneration. Cognitive dysfunction for patients with amnesic MCI progresses to AD at a rate of 12-15\% annually compared to 1-2\% for cognitively normal adults . Thus imaging-based tracking of amyloid and tau staging (Braak Stages) helps identify individuals at highest risk. \cite{muellerAlzheimersDiseaseNeuroimaging2005}


\section{Research Significance}
Dementia poses a huge challenge for aged individuals , even more so with a very high percentage of aging population \cite{affairsWorldPopulationAgeing2020}. In the US , about 7.2 million individuals are living with AD as of 2025 , a number expected to double by 2060 ( 13.8 million ) . The disease additionally creates a massive economic burden on healthcare systems and caregiving costs , reaching 384 billion dollars , far exceeding other chronic conditions. These trends highlight the need for improved diagnostic capabilities and treatment strategies. \cite{AlzheimersAssociation2025} 

Understanding lifetime dementia risk is essential for prevention and resource planning. One study estimated a 42\% baseline risk from age 55 to 95 for developing dementia, with higher prevalence observed in women (45–60\%), Black adults, and carriers of the \textit{APOE} $\varepsilon4$ allele \cite{fangLifetimeRiskProjected2025}.

Another interesting aspect that can help prevention and enhance resource planning estimates is the understandning of lifetime dementia. In one study a 42\% baseline risk from age 55-95 was found , with higher prevalence in women (45–60\%), Black adults, and carriers of the \textit{APOE} $\varepsilon4$ allele \cite{fangLifetimeRiskProjected2025}.

Rougly 697 cases of dementia exist per 10.00 people over 50 as reported in \cite{princeRecentGlobalTrends2016} . From those 324 cases are of AD and 115 from vascular dementia. Also in this study it is mentioned that women present greater rates and that prevalence doubles every two years. As is understood population aging continues to be the most important cause for increased rates of dementia , despite trends not deviating significantly from historical statistics. 

In order to enhance patient care, clinical decision-making, and research, highly precise diagnostic techniques are necessary for dementia in general and Alzheimer's disease (AD) in particular. In order to diagnose AD-related brain disease earlier and develop more individualized treatment plans, neuroimaging is a key technique \cite{ferreiraNeuroimagingAlzheimersDisease2011}.

\section{Objective of the Review}
The goal of this review is to summarize existing work and provide an intuitive explanation of each step in the pipeline from the imaging tools to the algorithms used for classification, discuss existing image datasets available and highlight gaps for future research. 