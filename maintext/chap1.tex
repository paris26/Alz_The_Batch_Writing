\chapter{Introduction}

\section{Contextual Background}
The evolution of neuroimaging has transformed the diagnostic landscape for Alzheimer's disease. Multiple imaging techniques, including MRI, PET, and CT, are used to support clinical evaluation.

Structural Magnetic Resonance Imaging (MRI) excels at quantifying brain atrophy, particularly in the hippocampus, where volumetric reductions serve as predictors of progression from Mild Cognitive Impairment (MCI) to AD.

Positron Emission Tomography (PET), with amyloid- and tau-specific tracers, enables visualization of amyloid-$\beta$ (A$\beta$) plaques and tau pathology (for more on AD pathology, see Appendix~\ref{Appendix-1}). Multimodal approaches integrating multiple imaging modalities provide superior sensitivity and specificity for early diagnosis and longitudinal disease tracking. AI- and deep-learning–enhanced multimodal frameworks now fuse neuroimaging data with genetic, clinical, and neuropsychological variables, achieving diagnostic accuracies exceeding 95\% in diverse cohorts \cite{marquezNeuroimagingBiomarkersAlzheimers2019}.

Neuroimaging offers objective and quantifiable biomarkers that surpass the limitations of subjective clinical assessment, enabling detection of subtle microstructural changes. Evidence shows that multimodal imaging improves differential diagnosis by monitoring atrophy, hypometabolism, and connectivity loss, while guiding treatment evaluation—critical in an era of emerging disease-modifying therapies \cite{kantarciNeuroimagingAlzheimerDisease2003}.

Recent innovations in imaging technologies have prompted updated diagnostic criteria. The National Institute on Aging (NIA) and the Alzheimer’s Association now emphasize staging based on amyloid, tau, and neurodegeneration profiles. Core~1 biomarkers include low CSF A$\beta$42/40 ratios or positive amyloid PET, while Core~2 biomarkers (e.g., elevated tau PET or plasma p-tau217) track disease progression. Advanced MRI protocols such as quantitative susceptibility mapping and diffusion tensor imaging target entorhinal–hippocampal circuits, where early tau deposition and synaptic loss precede neurodegeneration by up to two decades \cite{kantarciNeuroimagingAlzheimerDisease2003}.

For an expanded overview of AD pathology, see Appendix~\ref{Appendix-1}.

AD pathology develops silently for 15–20 years, making preclinical identification crucial. Tau PET and plasma p-tau217 trace early tau spread and predict neurodegeneration. Amnestic MCI—characterized by episodic memory impairment with preserved daily functioning—progresses to AD at 12–15\% annually, compared to 1–2\% in cognitively normal adults. Imaging-based tracking of amyloid burden and tau staging (Braak stages) helps identify individuals at highest risk \cite{muellerAlzheimersDiseaseNeuroimaging2005b}.


\section{Research Significance}
Dementia poses a profound public health challenge in aging populations. In the United States alone, 7.2 million individuals are living with Alzheimer's dementia as of 2025, a figure projected to nearly double by 2060 to 13.8 million. The increasing number of affected individuals also creates a significant economic burden on healthcare systems and unpaid caregiving, reaching \$384 billion, far exceeding many other chronic conditions. These trends highlight the need for improved diagnostic and therapeutic strategies \cite{AlzheimersAssociation2025}.

Understanding lifetime dementia risk is essential for prevention and resource planning. One study estimated a 42\% baseline risk from age 55 to 95 for developing dementia, with higher prevalence observed in women (45–60\%), Black adults, and carriers of the \textit{APOE} $\varepsilon4$ allele \cite{fangLifetimeRiskProjected2025}.

A meta-analysis found approximately 697 cases per 10,000 individuals aged 50+ with dementia (95\% CI: 546–864). Alzheimer's disease accounted for 324 per 10,000 (95\% CI: 228–460), while vascular dementia occurred at 116 per 10,000 (95\% CI: 86–157). The study also reports that prevalence doubles every two years, with overall higher rates in women. These findings confirm stable age-specific dementia prevalence. Although global trends show no major deviation from historical data, population aging remains the primary driver of increasing dementia cases \cite{princeRecentGlobalTrends2016}.

Dementia in general—and Alzheimer's disease (AD) more specifically—requires highly accurate diagnostic methods to improve patient care, clinical decision-making, and research. Neuroimaging is a central tool for revealing AD-related brain pathology, enabling earlier detection and more personalized treatment strategies \cite{ferreiraNeuroimagingAlzheimersDisease2011a}.

\section{Objective of the Review}
The goal of this review is to summarize existing work and provide an intuitive explanation of each step in the pipeline from the imaging tools to the algorithms used for classification, discuss existing image datasets available and highlight gaps for future research. 