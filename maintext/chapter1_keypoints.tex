\chapter{Summary of Points in chapter 1}

\section{Contextual Background}
1. Alzheimer presents itself even up to 10-20 years earlier before cognitve decline symptoms. 
2. Traditional clinical diagnosis based on cognitive testing is only 90 percent accucate and only after symptoms start showing

-> Neuroimaing modalites can detect pathological changes non-invasively way before symptoms emerge.
-> Most drugs work at preclinical or early symptomatic stages 


- Neuroimaging proved that AD is a continuum. 
- People with  amyloid and tau markers show: 
		1. Accelerated declince over 4 years ( seen via PET) 
		2. Glymphatic System dysfunction ( correlated with amyloid burden ) 
		3.  unusual fMRI patterns linked to amyloid and tau pathology 
		

The above are essential for prevention trials ( to see if drugs work) and find out
who is susceptible. 


\subsection{Modalities}

MRI

- MRI works by detecting volumetric changes of neuronal loss. 
- hippocampal volume , medial atrophy , cortical thickness, ventricular enlargement. 

- most widely available modality 
- already exist automated volumetric scales 
- hippo volume with ventricular volume better accuracy
- Deep Learning 94-99 percent 

strengths
- high res, availability , non-invasice , volumetric excellent , standardizable. 
limit
- atrophy other diseases, no molecular specificity , not functional changes , require other modalities


PET


Amyloid PET
- radioactive tracers
- amyloid-b plaques
- shows that deposition occurs in preclinical stages 

what does it give us  ? 
-> detection of amyloid in normal indiv. 
- Prediction from MCI to AD 
- differential diagnosis
- Treatment eligibility determination
- therapy monitoring 


Tau PET

