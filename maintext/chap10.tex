\chapter{Conclusion}

The field of research in Dementia and Alzheimer Classification is clearly defined by the Deep Learning Revolution. Most of the research has translated into integrating Deep Learning into multiple steps of the machine learning pipeline to maximize classification accuracy and significant gains in diagnosis. Deep Learning networks are used throughout the pipeline, like in pre-processing (Intensity Normalization, Registration , Skull Stripping , Volumetric Difference Estimation, Denoising ) but also as a tool to classify between stages of neurodegeneration.

Additionally we have seen networks trained to learn a prior, like in denoising , to be turned into generative networks for image generation.

Moreover deep neural networks are classified as universal approximators , meaning that they can resemble a function given enough parameters. So based on the ability to show gains connected to the power equations of scaling laws we can conclude that the main bottleneck of the field is data scarcity.

As has been mentioned efforts in the field have generated datasets for research purposes but there is a lack of benchmark datasets to check model generalizability.

Moreover even though the research front can generate competent enough models to detect and classify different stages of dementia the lack of trust and explainability , behind the model's reasoning blocks the usage in clinical settings.

Although these problems exist , methods that can overcome the limitations of data scarcity such as image fusion or data augmentation and even synthetic data have been explored showing promising results.

Finally even though limitations exist , by accounting for the progress that has happened in the field and the ability of models to become better serving as a prior , along with the fact that models perform state-of-the-art in research, it is highly likely that progress and future research will produce innovation to overcome data scarcity and model generalization as well as generate trust between clinicians , patients and Artificial Intelligence.