\chapter{Key Image Datasets for Dementia and Alzheimer’s Disease}

\section{ADNI}

The Alzheimer’s Disease Neuroimaging Initiative (ADNI) is one of the largest and most significant research initiatives in the field of Alzheimer’s disease. The ADNI is an example of a public – private collaboration, funded by the National Institute on Aging, the National Institute of Biomedical Imaging and Bioengineering (NIBIB), major pharmaceuticals such as Pfizer, Eli Lilly, Merck, GlaxoSmithKline and AstraZeneca, and non-profit organizations including the Alzheimer’s Association.

The ADNI has been established to develop and validate biomarkers, and also to validate neuroimaging markers for detecting cognitive and functional decline in the elderly population, as well as those suffering from Mild Cognitive Impairment (MCI), and cognitively normal controls. The ADNI will collect data that are related to genetics, cerebrospinal fluid (CSF), clinical information, cognitive function, and a wide variety of neuroimaging techniques including, but not limited to, structural MRI, FDG-PET, amyloid PET, tau PET, and Diffusion Tensor Imaging (DTI).

This will allow researchers to have access to a diverse set of data regarding changes in brain structure and changes in cognitive function. The primary focus of the ADNI is to accelerate work in early detection of Alzheimer's disease, track the progression of the disease, and evaluate the effectiveness of treatments.

The ultimate goal of the ADNI is to create standardized MRI/PET imaging datasets from large cohorts at various centers across the U.S., and Canada, to identify sensitive biomarkers for the disease, and to determine the efficacy of potential therapeutic interventions.

 One other major attribute of ADNI is the establshment of a publicly available data repository containing all their collected longitudinal information on clinical outcomes , biomarkers , imaging and cognition. With its ability to provide a substantial body of data and a better understanding of different biomarkers , it has become a crucially important part of AD research. \cite{muellerWaysEarlyDiagnosis2005}.

\section{AIBL}

The Australian Imaging, Biomarkers and Lifestyle (AIBL) study is an initiative to collect samples from 1000 individuals over 60 to collect data for AD research. The volunteers completed cognitive assesments , provided blood samples and went through detailed health and lifestyle questionnaires. One quarter of the participants received PET imaging while 10\% received ActiGraph activity monitoring and body composition scanning. 

In total the study collected data from 211 patients with AD, 133 participants with MCI, and 768 healthy controls. This dataset adds great value to ongoing AD research . The participants are reassesed at 18-month intervals to determine the predictive utility of available biomarkers. This longitudinal dataset is of immense value for tracking disease progression , evaluating predictive models and establishing clear labesl for different disease stages. Additionally it helps to define the individual variability in both pathological and normal cognitive aging and to highlight the importance of such studies. One of the biggest points is that it also provides insights into the stage of MCI which is crucial as has been mentioned for early detection and targeted intervention. \cite{ellisAustralianImagingBiomarkers2009}.

\section{OASIS}

The Open Access Series of Imaging studies (OASIS) is also a longitudinal imaging initiative simlar to AIBL. It consists of MRI and PET images for 1098 participants collected over a period of 15 years. The age of participants ranges from 42 to 95 with 605 cognitively healthy individuals and 493 with cognitive decline . 

The OASIS-3 Dataset consists of over 2000 MRI sessions featuring both structural and functional sequences, along with around 1500 raw PET scans. The dataset also includes additional products such as volumetric MRI segmentations and PET-derived metrics. Futhermore , OASiS provides data as APOE genotype infromation , and longitudinal clinical and cognitive assessments , making it a really valuable resource for research on AD and Dementia. \cite{marcusOpenAccessSeries2010}


