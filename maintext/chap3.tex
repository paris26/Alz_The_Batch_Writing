\chapter{Key Image Datasets for Dementia and Alzheimer’s Disease}

\section{ADNI}

The Alzheimer’s Disease Neuroimaging Initiative (ADNI) is one of the most extensive and significant research programs in the field of Alzheimer’s disease. ADNI represents a rare public--private partnership, supported by the National Institute on Aging, the National Institute of Biomedical Imaging and Bioengineering (NIBIB) of the NIH, major pharmaceutical companies (such as Pfizer, Eli Lilly, Merck, GlaxoSmithKline, and AstraZeneca), and nonprofit organizations including the Alzheimer’s Association.

The purpose of ADNI, which spans multiple clinical sites across the United States and Canada, is to identify and validate biomarker-based and neuroimaging indicators linked to cognitive and functional decline in aging populations, including individuals with Alzheimer’s disease (AD), mild cognitive impairment (MCI), and cognitively healthy controls. ADNI integrates genetic, cerebrospinal fluid (CSF), clinical, and cognitive data with a wide range of imaging modalities, including structural MRI, FDG-PET, amyloid PET, tau PET, and DTI. This multimodal dataset allows researchers to monitor changes in brain anatomy, metabolism, and cognitive performance over time for early diagnosis, disease monitoring, and treatment evaluation.

The main objectives of ADNI include the standardization of imaging procedures for long-term, multicenter PET and MRI data collection, the acquisition and validation of biomarkers and clinical metrics in a large participant cohort, and the identification of the most accurate and sensitive markers for diagnosing AD and MCI and evaluating treatment effectiveness. Another major goal is the establishment of a publicly accessible data repository containing comprehensive longitudinal information on clinical outcomes, biomarkers, imaging, and cognition. With its ability to power thousands of studies and substantially advance our understanding of neurodegenerative disease progression, ADNI has become a cornerstone resource for Alzheimer’s research worldwide~\cite{muellerWaysEarlyDiagnosis2005}.

\section{AIBL}

The Australian Imaging, Biomarkers and Lifestyle (AIBL) study is another major initiative designed to collect data from approximately 1000 individuals aged over 60 to enhance ongoing research on Alzheimer’s disease. In this study, volunteers participated in a screening interview, underwent cognitive assessment, provided 80\,ml of blood, and completed detailed health and lifestyle questionnaires. One quarter of the participants also underwent amyloid PET imaging and MRI scans, while a smaller subgroup of about 10\% received ActiGraph activity monitoring and body composition scanning.

In total, the study collected data from 211 patients with clinically confirmed Alzheimer’s disease, 133 participants with mild cognitive impairment, and 768 cognitively healthy controls. This final cohort represents a highly motivated and well-characterized group that contributes significantly to ongoing Alzheimer’s research. Participants are reassessed at 18-month intervals to determine the predictive utility of biomarkers, cognitive measures, and lifestyle factors as indicators of AD and future cognitive decline. This longitudinal dataset is invaluable for evaluating disease progression, informing predictive models, and establishing clearer labels for different disease stages. Moreover, it highlights the importance of large-scale studies for defining individual variability in both pathological and normal cognitive aging. It also provides critical insights into the MCI stage, which is widely considered one of the most important phases for early detection and targeted intervention~\cite{ellisAustralianImagingBiomarkers2009}.

\section{OASIS}

The Open Access Series of Imaging Studies (OASIS) is a longitudinal neuroimaging initiative similar in purpose to AIBL. It consists of MRI and PET imaging along with relevant clinical data for 1098 participants collected over a 15-year period. Participants range in age from 42 to 95 years, including 605 cognitively healthy adults and 493 individuals across different stages of cognitive decline.

The OASIS-3 dataset includes over 2000 MRI sessions featuring multiple structural and functional imaging sequences, as well as approximately 1500 raw PET scans and their corresponding post-processed outputs generated through the PET Unified Pipeline (PUP). The dataset also contains additional processed neuroimaging products such as volumetric MRI segmentations and PET-derived metrics. Furthermore, OASIS provides dementia status, APOE genotype information, and longitudinal clinical and cognitive assessments, making it an invaluable resource for research on aging and neurodegenerative disease~\cite{marcusOpenAccessSeries2010}.
