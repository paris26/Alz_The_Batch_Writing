\chapter{Key Image Datasets for Dementia and Alzheimer’s Disease}

\section{ADNI}

The Alzheimer’s Disease Neuroimaging Initiative (ADNI) is one of the most extensive and significant research programs in the field of Alzheimer’s disease. ADNI represents a rare public--private partnership, supported by the National Institute on Aging, the National Institute of Biomedical Imaging and Bioengineering (NIBIB) of the NIH, major pharmaceutical companies (such as Pfizer, Eli Lilly, Merck, GlaxoSmithKline, and AstraZeneca), and nonprofit organizations including the Alzheimer’s Association.

The purpose of ADNI, which spans multiple clinical sites across the United States and Canada, is to identify and validate biomarker-based and neuroimaging indicators linked to cognitive and functional decline in aging populations, including individuals with Alzheimer’s disease (AD), mild cognitive impairment (MCI), and cognitively healthy controls. ADNI integrates genetic, cerebrospinal fluid (CSF), clinical, and cognitive data with a wide range of imaging modalities, including structural MRI, FDG-PET, amyloid PET, tau PET, and DTI. 

The multiple modalities of the dataset  in order to provide researchers with data that show brain structure changes and cognitive abilities. It is used to accelearate work on early diagnosis , follow disease progression, and assess treatment efficacy.  

The goal of ADNI is to provide standardized datasets with MRI and PET imaging from large cohorts across mulitple centers , and identify sensitive markers , as well as assessing disease efficiency . 
 
 One other major attribute of ADNI is the establshment of a publicly available data repository containing all their collected longitudinal information on clinical outcomes , biomarkers , imaging and cognition. With its ability to provide a substantial body of data and a better understanding of different biomarkers , it has become a crucially important part of AD research. \cite{muellerWaysEarlyDiagnosis2005}.

\section{AIBL}

The Australian Imaging, Biomarkers and Lifestyle (AIBL) study is an initiative to collect samples from 1000 individuals over 60 to collect data for AD research. The volunteers completed cognitive assesments , provided blood samples and went through detailed health and lifestyle questionnaires. One quarter of the participants received PET imaging while 10\% received ActiGraph activity monitoring and body composition scanning. 

In total the study collected data from 211 patients with AD, 133 participants with MCI, and 768 healthy controls. This dataset adds great value to ongoing AD research . The participants are reassesed at 18-month intervals to determine the predictive utility of available biomarkers. This longitudinal dataset is of immense value for tracking disease progression , evaluating predictive models and establishing clear labesl for different disease stages. Additionally it helps to define the individual variability in both pathological and normal cognitive aging and to highlight the importance of such studies. One of the biggest points is that it also provides insights into the stage of MCI which is crucial as has been mentioned for early detection and targeted intervention. \cite{ellisAustralianImagingBiomarkers2009}.

\section{OASIS}

The Open Access Series of Imaging studies (OASIS) is also a longitudinal imaging initiative simlar to AIBL. It consists of MRI and PET images for 1098 participants collected over a period of 15 years. The age of participants ranges from 42 to 95 with 605 cognitively healthy individuals and 493 with cognitive decline . 

The OASIS-3 Dataset consists of over 2000 MRI sessions featuring both structural and functional sequences, along with around 1500 raw PET scans. The dataset also includes additional products such as volumetric MRI segmentations and PET-derived metrics. Futhermore , OASiS provides data as APOE genotype infromation , and longitudinal clinical and cognitive assessments , making it a really valuable resource for research on AD and Dementia. \cite{marcusOpenAccessSeries2010}


